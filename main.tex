% This is samplepaper.tex, a sample chapter demonstrating the
% LLNCS macro package for Springer Computer Science proceedings;
% Version 2.21 of 2022/01/12
%
\documentclass[runningheads]{llncs}
%
\usepackage[T1]{fontenc}
% T1 fonts will be used to generate the final print and online PDFs,
% so please use T1 fonts in your manuscript whenever possible.
% Other font encondings may result in incorrect characters.
%
\usepackage{graphicx}
% Used for displaying a sample figure. If possible, figure files should
% be included in EPS format.
%
% If you use the hyperref package, please uncomment the following two lines
% to display URLs in blue roman font according to Springer's eBook style:
\usepackage{hyperref}
\usepackage{color}
\renewcommand\UrlFont{\color{blue}\rmfamily}
%%%%%%%%%%%%%%%%%%%%%%%%%%%%%%%%%%%%%%%%%%%%%%%%%%%%%%%%%%%%%%%%%%%%%%%%%%%%%%%%
% Custom packages:
\usepackage[inline]{enumitem}
\usepackage[noend]{algpseudocode}
\usepackage[super]{nth}
\usepackage[table]{xcolor}
\usepackage{algorithm}
\usepackage{amsfonts}
\usepackage{amsmath}
\usepackage{amssymb}
\usepackage{booktabs}
\usepackage{csquotes}
\usepackage{mathtools}
\usepackage{multirow}
\usepackage{pgfplots}
\usepackage{tikz}
\usepackage{siunitx}
\usepackage{diagbox}
\usepackage[caption=false]{subfig}
\usepackage{svg}
\usepackage{tabularx}
\usepackage{xcolor}
\usepackage{caption}
\usepackage{wrapfig}
%%%%%%%%%%%%%%%%%%%%%%%%%%%%%%%%%%%%%%%%%%%%%%%%%%%%%%%%%%%%%%%%%%%%%%%%%%%%%%%%
% Prevent hyphenating
\fussy
\sloppy
%
\begin{document}
%
\title{Fast label-flipping attack for tabular data}
\titlerunning{Fast label-flipping attack}
%
%\titlerunning{Abbreviated paper title}
% If the paper title is too long for the running head, you can set
% an abbreviated paper title here
%
\author{
    Xinglong Chang\inst{1} \and
    Gillian Dobbie\inst{1} \and
    J\"org Wicker\inst{1}
}
%
\authorrunning{X. Chang et al.}
% First names are abbreviated in the running head.
% If there are more than two authors, 'et al.' is used.
%
\institute{
    The University of Auckland, Auckland, New Zealand \\
    \email{xcha011@aucklanduni.ac.nz, \{g.dobbie, j.wicker\}@auckland.ac.nz}
}
%
\maketitle              % typeset the header of the contribution
%
\begin{abstract}
The abstract should briefly summarize the contents of the paper in
150--250 words.

% \keywords{
%     First keyword  \and 
%     Second keyword \and 
%     Another keyword.
% }
\end{abstract}
%
%
%
\section{Introduction}
Hello world \cite{ref_article1}!

\section{Conclusion}
Hello world!

%
% ---- Bibliography ----
%
% BibTeX users should specify bibliography style 'splncs04'.
% References will then be sorted and formatted in the correct style.
%
\bibliographystyle{splncs04}
\bibliography{references}
%
\end{document}
